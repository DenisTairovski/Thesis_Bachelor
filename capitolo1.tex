\chapter{Introduction}
\label{sec:Introduction}
During these last 20 years, the technology had a massive improvement. Cellular communication first started with smart-phones, which were an absolute revolution in terms of portable communication, User-Interface, portability, etc. And, along with the smart-phone, new standard of communication had to be made in order to meet the user needs in terms of Data-Transfer, Reliability and Quality of Service. After the massive success of 1G, deployed in 1979, and 2G, in 1991, 3G, 2011, was a huge advancement in terms of download speed. Users had the opportunity to surf the net and stream music, which was impossible with 2G, even though it had the same capabilities but not the same speed. It wasn't until 2007, the year the first iPhone came out, that the demand for faster and more reliable network capabilities put ISPs under pressure. Not long after that, in 2009, 4G was finally deployed and, needless to say, since it's even used nowadays, in 2022, it was, and it still is, a huge success. Speed, Reliability, Quality. All those factor had to be taken in account for explaining the huge impact this standard of communication had brought into the telecommunication world. For over 10 years, 4G is still the most common standard used in smart-phones. The only improvement it needs is faster speed. Since the technology is evolving by the second, faster data transfer is an irremovable requirement. That's why 5G is the most valid option. With 20 times the average speed of 4G, and an incredibly low latency, it should be everyone is waiting for. But it comes with a huge downside: it's frequency. Since it works with much higher frequencies, between 30 GHz and 300 GHz, its wavelength is much smaller, and since it's been considered a urban environment, it's really difficult to have Reliability and Quality of Service compared to those of 4G. That's why the developing and deploying of Meta-Surfaces is needed. Let's explain what Meta-Surfaces actually are.
\section{What are Meta-Surfaces?}
\label{sec:what-meta-surface}
Meta-Surfaces(MS) is a device that allow to control the wireless propagation environment via an array of passive or reconfigurable reflecting elements. With the continuous improving of the urban environment, 4G and 5G antennas have a much harder job. ISPs and antennas manufacturers tried some ways to overcome the problem, like transmitting more power, using new frequency bands or installing more Base-Station (BS). But all those "apparent solutions" are not fitted for the job, since the need for less power consumption and budget-friendly solution is needed. A possible countermeasure is to use the problem for find a solution. Using the environment to improve the signal emitted from the BS is what MSs are for. MSs are simply tiles of reflective material that helps with the coverage of the signal emitted from a BS. Since it is going to be used a lot in urban environment, the need to be invisible or not obstructive to the eye is needed. It could be a panel or a paint appliable on the facade of a building. And, if more MSs are combined in a urban environment, what could be achieved is a Smart-Electromagnetic-Environment (SEME or SEE). MSs are split in two categories: Passive or Re-configurable. Passive MSs, also known as Electromagnetic Skins (EMS) or Intelligent Reflective Surfaces (IRS), are essentially pieces of reflective material that are placed on the facades of buildings. Re-configurable MSs, also known as Re-Configurable Intelligent Reflective Surfaces (RIRS), are reflective surfaces with the ability to actively adjust the reflective angle for better coverage. \\
Let’s examine both of these surfaces.
\section{Passive Meta-Surfaces}
\label{sec:passive-meta-surfaces}
As previously said, Passive Meta-Surfaces (PMS) are, essentially, reflective Surfaces placed, usually, on facades of building to improve coverage of the BS. They are composed by capacitive elements located ad sub-wavelength distance from a metallic plane. Each cell is loaded with active components in order to control the phase and the amplitude of the reflection coefficient. In these last years, this technology has emerged as a promising solution to improve wireless coverage, thanks to the fact that the amplitude and/or phase shift are tunable at each of its large number of reflecting elements. But the most crucial point is the ability to operate without power amplifiers. Each unit cell can be tuned such that signals bouncing off a RIS are combined constructively to increase signal in the position of the intended receiver or destructively to avoid leaking signals to undesired receivers. Another focus point it’s its mechanical design: since it’s light weight, it can easily be coated with the environment. But, even though it’s really easy to configure and install, it has a big downside: passive-IRS aided systems may be constrained by its high path-loss. Many solutions have been suggested, such as installing a large number of passive reflecting elements, deploying the passive-IRS near the BS. But those solutions have been proved inadequate for the purpose. Even though IRSs can be equipped with radio-frequency switches, this solution it’s economically impractical. That’s why, in some cases, active-IRSs are needed. 
\section{Active Meta-Surfaces}
\label{sec:active-meta-surfaces}
As explained before, passive Meta-Surfaces, are inconvenient in some cases. Since the environment could be filled with building or have a lot of open areas, there are some cases when moving the Meta-Surface is physically impossible, and its passive power amplification and phase shift are not enough to improve the quality of the signal. That’s why, in some cases, the deployment of active-IRSs in necessary. IPSs have to consider the hardware and energy cost, which in most cases is a massive downside.